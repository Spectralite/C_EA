\documentclass[11pt, a4paper]{article}

\usepackage{graphicx}
\usepackage{mathpazo}
\usepackage{changepage}

\usepackage{amsmath}
\everymath{\displaystyle}
\usepackage{mathtools}
\DeclarePairedDelimiter{\floor}{\lfloor}{\rfloor} 
\newcommand{\dd}{\mathrm{d}}

\usepackage[margin=1in]{geometry}
\usepackage{fancyhdr}

\pagestyle{fancy}
\fancyhf{}
\rhead{The UK University Integration Bee 2021/22}
\lhead{Round Two Problems - Group Round}
\rfoot{Page \thepage}

\begin{document}
    
% Title Page

\thispagestyle{empty}

\begin{center}
    \Huge\textsc{THE UK UNIVERSITY INTEGRATION BEE}

    \Huge 2021/22
    $$
    \int
    $$
\end{center}

\vspace{0.8in}

\begin{center}
    \Huge\sffamily\textbf{ROUND TWO}\\
    \LARGE\textbf{Saturday, 20 November 2021}
\end{center}

\vspace{0.7in}

\begin{center}
    \begin{adjustwidth}{2cm}{2cm}

\includegraphics[height=0.7in]{oxford-logo}
\hfill
\includegraphics[height=0.7in]{durham-logo}
\hfill
\includegraphics[height=0.7in]{warwick-logo}


\vspace{0.2in}

\includegraphics[height=0.5in]{imperial-logo}
\hfill
\includegraphics[height=0.5in]{cambridge-logo}

\end{adjustwidth}
\end{center}


\vspace{0.6in}

\begin{center}
    \Large
\textbf{Sponsored by}

\vspace{0.15in}

\includegraphics[width=0.5\textwidth]{jane-street-logo}
\end{center}
\clearpage


% Actual Questions


%Good sources: 

% DVKT Math - R1, Group & Shuttle
% Solving together - ^  
% Related instagram pages
% MIT Int bee - Crossword & shuttle
% Berkeley bee - used this last year - mainly R1
% Belarusian open math olympiad - R1
% Books like III and Almost impossible - mainly R1
% My brain - any

%Rally ones will be quite hard to do but maybe we could do a diverging one

%correct all log to ln at the end and recheck tally

%Lambert W integral? 

%Need some Weierstrass shite maybe with a dutis variant too

%These seem like good Round 1 or Round 2 part 1 things

%Tally of difficulty
%Easy: III (Q14,23,26)
% Mid-low: IIIIIIIIII (Q1,4,8,10,20,21,22,25,28,32)
% Mid: IIIIIIIII (Q3,11,12,18,19,27,30,35,36)
% Mid high: IIIIIII (Q2,15,16,29,34,39,40)
% Hard: IIIIII (Q5,6,7,13,17,33,38)
% Very hard: III (Q9,24,31)

%Q37 maybe i should remove so above doesnt include

%Count of problem type:

%Definite:
%Series: Q1,Q4, Q7, Q9,17,18,24,27.28,31
%-Catalan: Q17,24
%DUTIS: Q5,21,30,33,40
%Integral swap: Q34
%Reflection: Q6,11,12,16,17,20,25,26.29,35
%Contour: Q7,13 (some contour problems can be done in other ways - only one of these I only know a contour solution)
%Weierstrass: Q3,30,36,37
%Gamma: Q16,38
%Beta: Q38,
%Floor: Q4.9,28
%Integral of inverse: 32
%Hard af: Q31 (the tan(x^2) one)
%Smart trick: Q6, Q8,11,19,22,35,36,39
%Recurrence: Q3

%Indefinite: Q2, Q10,14,15,22,26


%Numbers left to put in: 69 - shuttle

%need more trig subs

% Start out with something friendly and nothing too hard till a couple problems in ! 

%Possible http://www.12000.org/my_notes/ten_hard_integrals/inse1.htm#x4-30002.1
% \begin{twocolumn}
\begin{enumerate}

%check all these problems!

    \item $\int_0^1 \arctan^2(x) \dd x$
%Possible but kinda grindy for 30 min
%Method: Integrate by parts, split integral and sub in tan into both . by parts on the first, you get log cos from 0 to pi/4. then you can consider log tan from 0 to pi/4, sub in tan, sub in u=e^x and then use geometric series to solve it. 
%Answer: pi^2/16 + pi/4 log2 - G
%Difficulty: Hard
\item $\int_2^{\infty} \ln\left(\frac{\floor{x}^2-1}{\floor{x}^2}\right) \dd x$
%check this on wolfram, it should turn into the product n^2-1/n^2 which is a neat little donny.
%mid low
%use a^n cos(ntheta) sum maybe ?
\item  $\int_0^{\pi} \frac{\dd x}{(2+\cos(x))^3} $
%Problem 7 from here : https://brilliant.org/discussions/thread/brilliant-integration-contest-season-1/
%Difficulty: Medium high
%Answer: pi/2root(3)
%This is kinda nice, maybe put it in round 1
% I want this but I don't want too many Weierstrass subs in one year, maybe it'd be good next year? I should decide which Weierstrass ones I want
\item $\int_0^{\pi} \ln(\sin{x})\cos^3(x) \dd x$
%This is just 0 by reflection but a bit harder to spot 
%changed to cos^3 as otherwise u=sinx gives a cheap solution
%Difficulty: Mid low
\item $\int \frac{t^2+1}{t^4+1} \dd t$
%https://www.quora.com/How-do-I-calculate-displaystyle-int_0-1-frac-arctan-x-2-1+x-2-mathrm-dx/answer/Brian-Sittinger
%Difficulty: Mid-low
%this is decent
%answer: 1/sqrt(2) arctan(t-t^-1/sqrt(2))
\item $\int_{-\infty}^{\infty} \frac{\cos(ax^2)-\sin(ax^2)}{1+x^4} \dd x$
%Everything from here: https://brilliant.org/discussions/thread/brilliant-integration-contest-season-1-part-3/
%Difficulty: Mid high
\item $\int_0^{\infty} \frac{\arctan(x)}{x^{\ln{x}+1}} \dd x$
%Mid low - use 1/x sub and it becomes Gaussian. Answer pi^{3/2}/4
%Originally integral 1 on round 1 lmao my first integral choice was shit    ; O anyway it is fair to assume Gaussian for this. 
\item $\int_0^{\frac{\pi}{4}} \sqrt{\tan{x}}\sqrt{1-\tan{x}} \dd x$
%Method: Sub x=root(tan) and sinx  and partial fractions, some other subs too.
%Difficulty: Hard
%https://www.instagram.com/p/B_AYH5TDVrx/
%from insta im p sure lmao
%Answer: pi \frac{sqrt{2+\sqrt{2}}{2}-pi
\item $\int_0^{\pi} \frac{x\sin{x}}{3+\cos^{2}{x}} \dd x$
%III page 153, check with wolfram
%expand as geometric series on denominator, nice
\item $\int_0^{\infty} \sin(x^2) \dd x$
%Fresnel!
%Answer sqrt(pi/2)
    
    \end{enumerate}
% \end{twocolumn}

\end{document}