\documentclass[11pt, a4paper]{article}

\usepackage{graphicx}
\usepackage{mathpazo}
\usepackage{changepage}

\usepackage{amsmath}
\everymath{\displaystyle}
\usepackage{mathtools}
\DeclarePairedDelimiter{\floor}{\lfloor}{\rfloor} 
\newcommand{\dd}{\mathrm{d}}

\usepackage[margin=1in]{geometry}
\usepackage{fancyhdr}

\pagestyle{fancy}
\fancyhf{}
\rhead{The UK University Integration Bee 2021/22}
\lhead{Round Two - Shuttle Round}
\rfoot{Page \thepage}

\begin{document}
    
% Title Page

\thispagestyle{empty}

\begin{center}
    \Huge\textsc{THE UK UNIVERSITY INTEGRATION BEE}

    \Huge 2021/22
    $$
    \int
    $$
\end{center}

\vspace{0.8in}

\begin{center}
    \Huge\sffamily\textbf{ROUND TWO}\\
    \LARGE\textbf{Saturday, 20 November 2021}
\end{center}

\vspace{0.7in}

\begin{center}
    \begin{adjustwidth}{2cm}{2cm}

\includegraphics[height=0.7in]{oxford-logo}
\hfill
\includegraphics[height=0.7in]{durham-logo}
\hfill
\includegraphics[height=0.7in]{warwick-logo}


\vspace{0.2in}

\includegraphics[height=0.5in]{imperial-logo}
\hfill
\includegraphics[height=0.5in]{cambridge-logo}

\end{adjustwidth}
\end{center}


\vspace{0.6in}

\begin{center}
    \Large
\textbf{Sponsored by}

\vspace{0.15in}

\includegraphics[width=0.5\textwidth]{jane-street-logo}
\end{center}
\clearpage


% Actual Questions


%Good sources: 

% DVKT Math - R1, Group & Shuttle
% Solving together - ^  
% Related instagram pages
% MIT Int bee - Crossword & shuttle
% Berkeley bee - used this last year - mainly R1
% Belarusian open math olympiad - R1
% Books like III and Almost impossible - mainly R1
% My brain - any

%Rally ones will be quite hard to do but maybe we could do a diverging one

%correct all log to ln at the end and recheck tally

%Lambert W integral? 

%Need some Weierstrass shite maybe with a dutis variant too

%These seem like good Round 1 or Round 2 part 1 things

%Tally of difficulty
%Easy: III (Q14,23,26)
% Mid-low: IIIIIIIIII (Q1,4,8,10,20,21,22,25,28,32)
% Mid: IIIIIIIII (Q3,11,12,18,19,27,30,35,36)
% Mid high: IIIIIII (Q2,15,16,29,34,39,40)
% Hard: IIIIII (Q5,6,7,13,17,33,38)
% Very hard: III (Q9,24,31)

%Q37 maybe i should remove so above doesnt include

%Count of problem type:

%Definite:
%Series: Q1,Q4, Q7, Q9,17,18,24,27.28,31
%-Catalan: Q17,24
%DUTIS: Q5,21,30,33,40
%Integral swap: Q34
%Reflection: Q6,11,12,16,17,20,25,26.29,35
%Contour: Q7,13 (some contour problems can be done in other ways - only one of these I only know a contour solution)
%Weierstrass: Q3,30,36,37
%Gamma: Q16,38
%Beta: Q38,
%Floor: Q4.9,28
%Integral of inverse: 32
%Hard af: Q31 (the tan(x^2) one)
%Smart trick: Q6, Q8,11,19,22,35,36,39
%Recurrence: Q3

%Indefinite: Q2, Q10,14,15,22,26


%Numbers left to put in: 69 - shuttle

%need more trig subs

% Start out with something friendly and nothing too hard till a couple problems in ! 

%Possible http://www.12000.org/my_notes/ten_hard_integrals/inse1.htm#x4-30002.1
% \begin{twocolumn}
\begin{enumerate}
    %Need 16 questions 

%How long to make this? Test with Archim members, 4-8 minutes?
%https://www.ukmt.org.uk/sites/default/files/ukmt/ptmr/ptmr-sample-shuttle.pdf

%Possible stuff
%Good sources: MIT Integral bee for short qns
%int sin x 0 to pi or something as a start as its easy
% int e^-ax sinx dx from 0 to inf.

%Some quite simple reflection thing for a first qn

%also want some question where you get a parameter which absolutely kills eg a sin0, cos0 or e^0 ? or even an ln0 to get a -infinity answer hahahahahah. WILL NEED TO ANNOUNCE -INFINITY or INFINITY IS VALID ANSWER

%techniques which are valid:

%Weierstrass for Q3,4
%Dutis for Q4
%series for Q2-4
%Reflection for 1-4
%Floor function
%Odd / even / 'adding zero trick'

%3. $\int_a^{-a} e^{\text{some quadratic}} \dd x$
%Complete square
%Could do something interesting like maximum value given some constraints and then force the answer to be pi.
%too easy to be here with infinity as the previous clue
%Next year maybe

\textbf{\LARGE Shuttle 1} \newline

\begin{flushright}
\textbf{\LARGE A1}
\end{flushright}

Evaluate $$\int_0^1 \ln^2(x) \dd x$$
Pass on the answer squared. \newline  \newline \newline \newline \newline \newline
%get 2
%pass on 4
%Something quite tame. Maybe an even/odd function thing to help simplify it to make it fairly quick.
%Give the answer squared, it'll be 4

2. $a$\textit{ is the number you will receive} \newline
Evaluate $$\int_0^{\frac{\pi}{2}} \frac{\sin(x)}{\sin(x+\frac{\pi}{a}
)} \dd x$$
Your answer will be of the form $\frac{\pi}{b}$, pass on $b$. \newline  \newline \newline \newline \newline \newline
%get pi/2sqrt(2)
%pass on 2sqrt(2)
%From MIT Integral bee 2019 or 2020 something
%https://www.wolframalpha.com/input/?i=integral+%28sin%28x%29%29%2F%28sin%28x%2Bpi%2F4%29%29+from+0+to+pi%2F2
%have the pi/4 be pi/a and pass it on, not possible without a but can do a little intermediate work
%Answer is of the form pi/csqrt(d), either pass or write c.

3. $a$\textit{ is the number you will receive} \newline Evaluate $$\int_0^{\frac{\pi}{2}} \frac{\dd x}{1+\tan^{a}(x)}$$
Pass on the 4 times the answer. \newline  \newline \newline \newline \newline \newline
%get pi/4
%pass on pi
%put in middle of shuttle so everyone is like woah we didnt need to pass the number !  Maybe the number passed to them can be 7945
%Answer pi/4, maybe modify so its integers only, like 1/pi on this ? and a+b
%integral 3

4. $a$\textit{ is the number you will receive} \newline Evaluate $$\int_0^{a} x\ln(\sin{x}) \dd x$$
%answer -1/2 pi^2 ln(2)
%Impossible to do unless it's up to pi so do this as an integral 4. 
% This is a bit of an easy shuttle but i guess to ease people in ?
\newpage
%Something fairly challenging. Can do one of the indefinites like one of the next year fam problems. 

Shuttle 2

1. $a$\textit{ is the number you will receive} \newline Evaluate $$\int_0^{2 \pi} \sin(\sin(x)-x) \dd x$$. Pass on the answer. \newline  \newline \newline \newline \newline \newline   %do this as integral 1
%Answer 0, pass it on
%neat reflection

%Something somewhat hard ngl. Maybe a clever sub + reflection? Not obvious the answer is 0, either do a weird integral or mask it by some weird question
2. $a$\textit{ is the number you will receive} \newline Evaluate $$b=\int_{a=0}^{\infty} \ln(1+\frac{169}{x^2}) \dd x$$. Pass on the second digit of $b$. \newline  \newline \newline \newline \newline \newline 
%III Page 45,46
%get 13pi
%pass on 0
%Answer is 13pi, maybe say what's the first digit of the answer so that it's 0.
%or take the irrational part? like a+bsqrt(c) and take rational one.

3. $a$\textit{ is the number you will receive} \newline Evaluate $$b=\int_0^{1} \frac{\ln(a=0(x+1))}{x^2+1} \dd x$$. Pass on your answer. \newline  \newline \newline \newline \newline \newline   %integral 2 or 3 ? 
%get a=0
%pass on -infinity (announce that -infinity, infinity are valid)
%Serret integral but SIKE it's just -infinity lmaooooooooo, take a=0


4. $a$\textit{ is the number you will receive} \newline Evaluate $$\int_{\frac{1}{a}}^{\infty} \frac{\ln{x}}{x^2+\pi^2} \dd x$$. 
%https://www.wolframalpha.com/input/?i=integral+%28ln%28x%29%29%2F%28x%5E2%2Bpi%5E2%29+from+0+to+infinity
%log(pi)/2
%Reasonable series one. Just need to figure out how to use -infinity here. 

%DSomething difficult ish, mid low or medium. Maybe a series problem with floor. (to design this pick a series and work backwards). A series I like is Basel, any Zeta thing, Catalan.
\newpage

Shuttle 3

1. $a$\textit{ is the number you will receive} \newline Evaluate $$b=\int_{-1}^1 \frac{\sin(\cot^{-1}{x})+\cos(\tan^{-1}{x})}{x^2+1} \dd x$$. Pass on $b^2-2$. \newline  \newline \newline \newline \newline \newline
%This is the same odd even idea but with a little sub thrown in too
%the sin part should be 0, the remaining part should be like 2cos(pi/4) = sqrt(2)
%get sqrt(2)
%pass on 0
%is this too easy?

2. $a$\textit{ is the number you will receive} \newline Evaluate $$b=\int_0^{a} x^3e^{-x} \dd x$$. Pass on $2-b$. \newline  \newline \newline \newline \newline \newline 
%get 0
%pass on 2
%Do an integral which ends up as 0 because you get the bounds to end up being of 0 length hahahahahahahahahah. Something of moderate difficulty, , , maybe a sin(x^2) so that the mfs cum over fresnel integral? or maybe some aids by parts stuff HAHA (or even troll the gamma mfs)
%maybe have this and then follow up with the ln(ax) one to be an ultimate troll and then have a fairly difficult integral at the end lmfaoooo, probably some dutis or series type thing.

%Just take the min(floor(this),1). That will look better

3. $a$\textit{ is the number you will receive} \newline Evaluate $$b=\int_0^{\infty} \frac{\ln(x^{\frac{1}{a}})}{x^{\frac{1}{a}}(x+1)^2} \dd x$$. Pass on the least integer greater than $|b|$. \newline  \newline \newline \newline \newline \newline  
%from:https://www.instagram.com/p/CBVbxIID7zN/
%solution: https://www.instagram.com/p/CBf1Jg3jfTX/
%get pi/2 out of it
%take absolute value
%pass on 2
%Possible idea is lnx/1-x^2 from 0 to 1 with 2sqrt(a); take the irrational part

4. $a$\textit{ is the number you will receive} \newline Evaluate $$b=\int_0^{\frac{\pi}{2}} \frac{\cos{\theta}}{a-\sin{2\theta}} \dd x$$.
%get pi/4
%https://www.wolframalpha.com/input/?i=integral+%28cos%28x%29%29%2F%282-sin%282x%29%29+from+0+to+pi%2F2
%Impossible to do unless it's up to pi so do this as an integral 4.  
\newpage
%Maybe swap shuttle 2 and 3?

Shuttle 4

1. $a$\textit{ is the number you will receive} \newline Evaluate $$b=\lim_{n \to \infty} \int_{\frac{5}{6}}^{\infty} e^{-x^n} \dd x$$. Your answer should be of the form $\frac{c}{d}$ for $c,d \in \mathbb{Z}$ are coprime, pass on the prime factors of $d$. \newline  \newline \newline \newline \newline \newline 
%get 1/6
%pass on 3,2

%Need to decide what number to use for this, maybe to get pi/6 and then say pass the prime factors of denominator

2.$a,b$\textit{ are the numbers you will receive where }$a>b$ \newline Evaluate $$c=\int_0^1 \frac{x^{a=3}-x^{b=2}}{\ln{x}} \dd x$$. Pass on the greatest integer less than $c$. \newline  \newline \newline \newline \newline \newline
%Given two numbers, a>b
%Answer is ln(4/3), should pass on 1
% I want to get two out of this, maybe do like the ceiling of this . . .or twice the ceiling?

3. $a$\textit{ is the number you will receive} \newline Evaluate $$b=\int_0^1 \frac{1-x}{x^a+(x^a+1)^a} \dd x$$. Your answer will be of the form $b=\frac{x}{y}(\ln(w)-z)$ where $x,y,z,w \in \mathbb{Z^+}$ and $x,y$ are coprime. Pass on $\frac{x+\sqrt{x+y}}{z}$. \newline  \newline \newline \newline \newline \newline 
%b=1
%https://www.wolframalpha.com/input/?i=integral+%281-x%29%2F%28x%2B%28x%2B1%29%29+from+0+to+1
% pass on 2

4.$a$\textit{ is the number you will receive} \newline Evaluate $$\int_0^{2\pi} \frac{x}{a - \cos^2{x}} \dd x$$. 
%Answer 2pi^2. Just quickly go through! 
%https://www.instagram.com/p/B-j-Ip4jnfg/

%Solution: https://www.instagram.com/p/CBDTZUrDkL1/
%Possible for rally and then pass onto something that uses phi like these:
% https://www.instagram.com/p/B-j-Ip4jnfg/ - a bit long
% https://www.instagram.com/p/B-mbVhUD-Ye/ - not really special use of phi, just a dutis problem
% https://www.instagram.com/p/B9ugM68JhJ7/ - decent length and not too difficult. 
% https://www.instagram.com/p/CB-A2ULD-8Z/ - answer is 1, need to work out method. Either this or the one above.
% Maybe final shuttle as its quite hard and long ?
    \end{enumerate}
% \end{twocolumn}

\end{document}